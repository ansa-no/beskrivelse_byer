\chapter{Jobb}

I Tyskland kan ikke lønna du får ved å jobbe som student ved siden av studier komme i nærheten av å sammenliknes med norske forhold, men det kan jo hende du finner en studierelatert jobb, eller bare kan tenke deg å ha litt avveksling, treffe noen hyggelige kollegaer, knytte kontakter og spe litt på studielånet.

Hva gjør du da? Trenger du arbeidstillatelse, skattekort etc.?

Norge er med i EØS og vi trenger derfor ikke arbeidstillatelse, men du må være registrert ved Einwohnermeldeamt i byen du bor i.
Du trenger et Lohnsteuerkarte, som du får etter å ha meldt deg inn i registeret hos Einwohnermeldeamt. Ditt første skattekort har Steuerklasse 1. På dette får du lov til å tjene frem til 400\euro{} uten å måtte betale skatt og avgifter (400\euro{}-Job - Geringfügigkeitsgrenze).


\begin{figure}[h]
\center
\includegraphics[width=0.31\textwidth]{./gfx/arena}
\caption{Den nye Allianz Arena, hjemmebanen til FC Bayern München}
\end{figure}

Du spør du deg kanskje hvorfor grensen er så lav, og dette er fordi det finnes en 20-timers-grense for studenter i Tyskland, så det skal unngås at studentene er mer på jobben enn de er på universitetet... Ferier og helger er utelatt fra denne 20-timers-grensen, så du regnet over hele året kan tjene 7235\euro{}. Det du tjener i Norge er ikke underlagt dette beløpet.
Har du to jobber må du ha et Lohnsteuerkarte til, og dette har da Steuerklasse 4. Her betaler du skatt, men får dette tilbakebetalt dersom du holder deg under den øvre grensen på 7235\euro{}.
Se disse sidene for mer informasjon: \\
\url{http://www.seezeit.com/opencms/export/download/Downloadgalerie_SozialesBeratung/Ich_und_meine_Rente.pdf} \\
\url{http://www.jobber.de/_1_myjobber/jobforum,arbeitsrecht.html}
