\chapter{Annen nyttig informasjon}

\section{Oppholdstillatelse}
Ettersom Norge er en del av Schengen og EØS, trenger vi nordmenn kun å ta oss en tur til utlendingskontoret i rådhuset (Kreisverwaltungsreferat, U-Bahn Poccistraße) hvor vi får et dokument som sier "Jeg har lov til å være hær så lenge jeg vil". 

Her er internettsidene:\\
\url{http://www.muenchen.de/rathaus/Stadtverwaltung/Kreisverwaltungsreferat.html}

Ta med pass! Man blir spurt en rekke spørsmål, bla. bostedsadresse og religion.
Ta vare på dokumente(t)ne!
Det er her man også melder om flytting, innad i byen, men også hvis man flytter tilbake til Norge.

\begin{figure}[h]
\center
\includegraphics[width=0.31\textwidth]{./gfx/lech}
\caption{Alltid masse nysnø i Lech}
\end{figure}

\section{Forsikring}
Dette er noe mange ikke tenker på. Spesielt ansvarsforsikring er noe en bør har. Hvis en enda er forsikret hjemme i Norge gjennom foreldre eller foresatte så er saken grei. Men, hvis en skulle være så uheldig å forårsake en ulykke hvor en blir erstatningspliktig, sliter en uten ansvarsforsikring. ANSA tilbyr en meget god forsikringspakke med alt en trenger (ansvars-, innbo-, reiseforsikring etc.). En kan også skaffe seg ansvarsforsikring i Tyskland. Her heter det "Hauptpflichtversicherung". Flere studier vil kreve ansvarforsikring, ved f.eks. jobbe på lab. ANSA forsikring dekker ikke slikt, da vil man trenger å oppsøke et forsikringsselskap og forklare til hva, man trenger forsikringen. 

Forsikringsselskap: 
huk24.de
Allianz


Flere banker driver også med forsikring.


\section{Telefoni}

I Tyskland er det nesten som i Norge når det kommer til mobiltelefoner. Man kan kjøpe mobil med eller uten abonnement. Hvor man kan kjøpe mobil varierer, i store elektrokjeder som Saturn (tilsvarer Elkjøp og Bonus i Norge) der mobiler eller tilknyttet ulike typer abonnement, eller små telefonbutikker som er tilknyttet et abonnement (f.eks hvis Telenor hadde hatt sin egen butikk hvor det ble solgt mobiltelefoner med kun Telenorabonnement).
Hva som lønner seg mellom abonnement og kontantkort er helt avhengig av hvor mye du kommer til å ringe og hvor/hvem du ringer.
Man trenger som oftest en tysk bankkonto for å få et tysk mobilabonnement.

Vanligvis er man tilknyttet et abonnement 12-24 mnd. Dette er ganske lenge, og man må derfor tenke nøye gjennom hvilket abonnement man velger og hos hvilken leverandør.
En annen mulighet man har som er ganske spesielt i Tyskland er at man kan få ett mobilnr og ett eget "homezone"-nr. Denne "Homezonen" gjelder i 1 km omkrets fra der du bor og når du befinner deg der, kan du ringe med fasttelefonpriser fra "homezone"-nr ditt.. Spør forhandleren for ytterligere informasjon.
Det finnes veldig, veldig mange leverandører, noen leverandører er kun å finne på internett, disse pleier å være rimelig. En kort oversikt følger her:

De dyre: T-mobile (veldig lik Telenor)
 
De rimelige: Vodafone, O2 - studentenes favoritt,  callmobile.de. 

Disse tre har studenttariffer og kan bli ganske rimelig. O2 har til og med ubindende abonnemnt, som heller ikke er så dyr.
 
Fulliste finner dere her:\\
\url{http://www.verivox.de/handytarife/calculator.aspx}
 

Kontantkort anbefales de 2 første månedene av oppholdet ditt i Tyskland.
Er enkelt å få tak i, men kan også være den dyreste varianten. Hvis en ikke ringer så mye, er dette det greieste. Callmobile er forhåndsbetalt (pre-paid) som også tilbyr fri surfing for 10 euro i måneden.



\section{Internett}

M-net og Alice er de største operatørene på fast ineternett. De fleste universiteter tilbyr gratis trådløst nettverk, som man kan koble seg opp på fra sin egen laptop.


En oversikt over ulike firmaer som tilbyr internett finner du på denne siden:
\url{http://www.finanztip.de/preislotse/sonstiges/internet.php?phpurl=internetanbieter.php}

\begin{figure}[h]
\center
\includegraphics[width=0.31\textwidth]{./gfx/spitzing}
\caption{Klar for skogskjøring i Spitzingsee}
\end{figure}

\section{Bank}

Noe av det første man må skaffe seg når man flytter til Tyskland er en tysk bankkonto. For det første tar de færreste butikkene visa-kort, til og med ikea. For det andre krever f.eks. mobiloperatørene at man har en tysk konto. I Tyskland er det vanlig å bruke noe som heter EC-kort, men det er viktig å huske på at mange små butikker ikke tar kort i det hele tatt.
Når man velger bank er det viktig å tenke på at det koster å ta ut penger i andre banker/minibanker enn den man selv tilhører. Det vil si at det lønner seg å velge en bank med mange terminaler eller en i nærheten av der man bor. Deutsche Bank og Sparkasse er to banker som pleier å ha mange filialer i de fleste byene, og det er også mulig å få egne studentkontoer hos dem.
Man kan ikke bare droppe innom å opprette en konto, man må først bestille en time med banken for så å få en bankkonto. Husk pass.



\section{Kollektivtrafikk} \label{mvv}

München har et meget bra kollektivtilbud. By-tog (S-Bahn), undergrunnstog (U-Bahn), trikk (Tram) og buss.

Månedskort på MVV (Münchner Verkehrs- und Tarifverbund) koster fra 30 til 80 \euro{} uten studentrabatt, avhengig av hvor stort reisenett du trenger. Som immatrikulert student får man en rabarert månedsbillet.
Dette er relativt dyrt i forhold til andre tyske byer, selv med den lille studentrabatten en får på månedskort.

Se MVV for info:\\
\url{http://www.mvv-muenchen.de/de/tickets-preise/tickets/schule-ausbildung-und-studium/}


\section{Toll}

Har man tenkt å ta meg seg noe spesielt (bil, hund etc) fra Tyskland til Norge eller omvendt, kan det være lurt å ta seg en titt på \href{http://www.toll.no}{www.toll.no} eller  \href{http://www.zoll.de}{www.zoll.de} først.

En kan også sjekke mulighetene for utføre varer momsfritt. Ettersom vi er studenter og som oftest kun har registrert en postadresse i utlandet, faller vi litt mellom to stoler. Vi kan lovlig få tilbake norsk moms på varer kjøpt i norge ved utførsel (som turister). Se \url{http://www.toll.no/templates_TAD/Topic.aspx?id=256215&epslanguage=no}. \\
Det greieste er alternativ 1: Print ut skjema, og fyll inn detaljene. Skjema med kvittering stemples hos tollvesenets skranke når en drar tilbake til Tyskland på flyplassen. Husk å gjøre dette før du sjekker inn bagasjen, hvis varen du skal utføre ligger der.\\
Orginal skjema med kvittering sender du så til butikken du kjøpte varen, påført ditt kontonummer, og du får tilbakeført merverdiavgiften.


En kan også gjøre dette når eg har kjøpt en vare i Tyskland, og reiser hjem til Norgei. Dette er muligens ikke helt etter boka, men går så lenge man sier at man kun har vært på besøk i Tyskland (ikke registrert adresse). Skjema er meget likt det norske: \url{http://www.zoll.de/SharedDocs/Downloads/DE/FormulareMerkblaetter/Privatpersonen/steuer_2004.pdf?__blob=publicationFile}. \\
På flyplassen sjekker man ikke inn bagasjen, men nevner at man skall innom tollvesenet. Tollvesenet må da se varen og stemple dokumentet samt kvittering før de sender inn bagasjen din fra deres eget transportbånd. \\
All informasjon finner du her: \url{http://www.zoll.de/DE/Privatpersonen/Reisen/Reisen-nach-Deutschland/Zoll-und-Steuern/Tax-free-einkaufen/tax-free-einkaufen_node.html}


